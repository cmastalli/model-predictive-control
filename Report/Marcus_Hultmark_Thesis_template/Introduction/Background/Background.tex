\section{Background}
\label{chap:Background}

The Mechatronics Research group at Simon Bolivar University in Caracas, Venezuela is focused on the development of solutions based on the integration of knowledge in the fields of Automatic Control, Computer Science, AI and Robotics, Electronics and Mechanics. This development is made on a project based strategy, with projects coming from both industry and academia. One of the projects that has been developed during the last years is the submarine of the university, named PoseiBot. The development of this submarine robotic platform has been made in several phases through the years. In the latest phase, a Model Predictive Control (MPC) strategy was implemented to control the submarine achieving good results in terms of the control effort, error reduction and robustness. MPC is an advanced control technique based on solving an optimal control problem with a finite prediction horizon in each sample. MPC is commonly applied to large systems with slow dynamics, but recently with the increase of computational power and the development of new algorithms that are more efficient, systems with faster dynamics are being targeted to be controlled by predictive methods. This was implemented through communication of the submarine's microcontroller to a remote computer out of the water using serial wired communication to a flotation device that communicates wirelessly with the remote computer. In the computer, signal acquisition  and data processing was done via LabVIEW \texttrademark  and MATLAB\textsuperscript{\textregistered}, respectively. \\

Another project developed in the group is the usage of helicopter models as a robotic platform for powerline inspection. Due to the wind conditions around the powerlines to be inspected, a robust control algorithm is required to assure a safe operation of the quadrotor while maneuvering around the lines. There is a strong interest on using instead the quadrotor available in the research group instead of the helicopter because of the increased stability. \\

Based on the requirements set by the aforementioned projects, there has been an increasing interest in advanced control techniques and specially MPC applications for both platforms. MPC has been proven as an efficient tool for solving multivariable control problems that might be difficult to decouple, in plants that might have some restrictions in some variables and might even be nonlinear. MPC can handle all of these requirements satisfactorily while being optimal in the solution, which is important in cases where resources are limited.\\




But in order to provide a solution that could fit both applications in a relatively quick way, some standardization and abstraction is required in the solution. That is where the benefits of using ROS apply, and it also represented an opportunity to extend the usage of this tool within the group. 
