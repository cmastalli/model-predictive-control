\section{Purpose}
\label{chap:Purpose}

The purpose of this project is the creation of a ROS package to implement MPC strategies in different platforms in a standard and abstract way. The standard characteristic is necessary to get a package that is easy to use without needing to know how it works internally. The abstraction required comes from the fact that the software must work equally good independently of the platform that is being controlled. Of course, there are limitations on how much abstraction can be obtained, since every application will require the development of a process model for the package to use. However, the goal is to use the properties of ROS to achieve this.\\

To reach this goal, the first activity to do will be an extensive bibliographic revision about MPC and its varieties, either theoretically and implemented in different systems. Another topic included in this revision is quadratic programming, since the interest is to apply MPC with constraints. When the MPC problem is not constrained the control law can be calculated exactly, but when constraints are added the solution must be obtained numerically, and there is when quadratic programs arise. \\

After this phase, the focus will be the design of the organization and development of the package. This is an important phase of the project because a proper design will allow a modular organization of the functionality, i.e. the nodes in the package will be enabled to be used in different combinations without altering the way the software works. The development is carried out in an iterative way, so the code can be tested and improved in each iteration. \\

The third phase consists in the creation of a demonstrative platform to use it as an overall test for the package. This includes the creation of a Model and Simulator classes for such system. The model used is kept simple to ease the validation of the results. The chosen system for this purpose is a water recirculation system with two tanks, that is used in the Automatic Control courses. This will save the modeling work, since this is a well known plant. \\

At this point, the MPC package will be already running properly, and then the time to try it in a relevant platform comes. The modeling of the quadrotor platform will be performed to use it with the MPC package and perform simulated tests in trajectories of interest. This phase may require several tests in order to characterize and obtain the properties of the quadrotor if there is no relevant work available about it. The model also requires a validation process for itself to prove that it works in an adequate manner. 
