\begin{abstract}

%The focus of this work is centered around Model Predictive Control (MPC) and its application. In this project, there are two main goals: firstly, is the development of a library that works as an infrastructure when building MPC applications; secondly, the application of MPC controllers to plants with fast dynamics, such as the quadrotor. Chapter 2 presents the mathematical background required to understand MPC, and also explains the key elements present in it. In Chapter 3, the focus is on the modeling of the quadrotor platform based on the physic laws that rule the dynamics of the drone. The model was tested with simple trajectories in several directions to verify its proper operation. Chapter 4 introduces a thorough description of the MPC software architecture, in order to give insight about how it works internally. The proposed architecture allows benefits proper of Object Oriented Programming (OOP), such as obtaining reusable code, a modular and easy-to-understand structure and the encapsulation of data, that removes the need to know the entire program to know how to use it. The library was tested on two numerical simulators developed as a part of this project: one of a water tank system in the Control Lab at Simon Bolivar Univerisity, and the simulator of the quadrotor itself. The simulation of the MPC controller applied to the quadrotor shows a good response when following different trajectories in the presence of simulated noise. However, a considerable offset from the trajectory is obtained, which demonstrates the considerable sensibility of the controller to the accuracy of the used model. Regarding timing, the controller requires a lot of computational power to implement in real-time with the current update frequency for the drone, or to cut the CPU-time via software and using the suboptimal solutions of the MPC.

Model Predictive Control is a receeding horizon control technique that is based on making predictions in the future for a determined number of steps, using a model of the system to be controlled. This thesis report is centered around Model Predictive Control (MPC) and its application. In this thesis, there are two main goals: firstly, is the development of a software structure that uses the properties of Object Oriented Programming (OOP) and the Robot Operative System (ROS) to ease the use of MPC applications. Secondly, the use and verification of the capabilities of MPC controllers in plants with fast dynamics, such as the quadrotor. A linearized model of the quadrotor is developed for the controller to perform the predictions, and the non-linear version is used to make a numerical simulator to test the application. The MPC controller shows a good response when following different trajectories in the presence of simulated noise. However, a considerable offset from the trajectory is obtained, which demonstrates the considerable sensibility of the controller to the accuracy of the used model.
\end{abstract}

