\begin{abstract}

The focus of this work is centered around Model Predictive Control (MPC) and its application. In this project, there are two main goals: first, is the development of a library that works as an infrastructure when building MPC applications; second, the application of MPC controllers to plants with fast dynamics, such as the quadrotor. Chapter 2 presents the mathematical background required to understand MPC, and also explains the key elements present in it. In Chapter 3, the focus is on the modeling of the quadrotor platform based on the physic laws that rule the dynamics of the drone. The model was tested with simple trajectories in several directions to verify its proper operation. Chapter 4 introduces a thorough description of the MPC library architecture, in order to give insight about how it works internally. The proposed architecture allows benefits proper of Object Oriented Programming (OOP), such as obtaining reusable code, a modular and easy-to-understand structure and the encapsulation of data, that removes the need to know the entire program to know how to use it. The library was tested on two numerical simulators developed as a part of this project: one of a water tank system in the Control Lab at Simon Bolivar Univerisity, and the simulator of the quadrotor itself. In both cases the controller presents good performance, but in the case of the quadrotor, requires a lot of computational power to implement it in real-time, or to cut the CPU time via software and using the suboptimal solutions of the MPC.
\end{abstract}

