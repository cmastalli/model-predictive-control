% !TeX spellcheck = sv_SE
\selectlanguage{swedish}

%\todo[inline]{avdnr etc??}

\noindent \begin{tabular*}{1.0\textwidth}{|@{} p{0.9835\textwidth}|}
\hline
\noindent \begin{tabular*}{1.0\textwidth}{p{0.97\textwidth}}
\textcolor{white}{.}\\[-10pt]
\end{tabular*}
\noindent \begin{tabular*}{1.0\textwidth}{p{0.24\textwidth} p{0.69\textwidth}}
\multirow{3}{*}{\includegraphics[scale=1]{./Abstract/KTH-pic}} & \begin{center}Examensarbete MMK 2014:90 MDA 465\end{center}\\[-20pt]

& \begin{center}\reportTitle\end{center}\\[-20pt]
& \begin{center}Rene Diaz \end{center}\\ 
\end{tabular*}
\noindent \begin{tabular*}{1.0\textwidth}{p{0.24\textwidth}|p{0.33\textwidth}|p{0.33\textwidth}}
\hline
{ \footnotesize Approved:} & { \footnotesize Examiner:} & { \footnotesize Supervisor:}\\
2014-11-14 & Lei Feng & Bengt Eriksson \\
\hline
& { \footnotesize Commissioner:} & { \footnotesize Contact person:}\\
&  Universidad Sim\'{o}n Bol\'{i}var & Carlos Mastalli \\ \hline
\end{tabular*}
\end{tabular*}\\

\justify
\textbf{Sammanfattning}\\

Att estimera ett batteris laddstatus �r av stor vikt n�r det g�ller att f�rl�nga livsl�ngden hos batteriet och att optimera urladdningscykler. I detta examensarbete utvecklades en hybridmetod f�r laddstatusestimering som sedan implementerades i en motorstyrning. Metoden anv�nder en kombination av impedansm�tningar och en linj�r regressionsmodell f�r estimering. M�tningar g�rs vid stillast�ende. Metoden kr�ver inga externa sensorer eller andra batterianslutningar �n terminalanslutningarna. Den slutgiltiga modellen uppvisar god prestanda men uppvisar ett visst temperaturberoende.

\end{abstract}