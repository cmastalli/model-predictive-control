\chapter{Results and Discussion}
\label{chap:Results_and_Discussion}

This chapter will be divided in three parts: in the first one, the settings and parameters to perform the tests will be presented, together with the achieved results. In the second part, an analysis of these results will be developed and the third part will be a final summary.

\section{Settings and Parameters}

It is important to mention that the MPC is not directly controlling the input signals going to the motors. The reason for this is that the model that is being used to perform the calculations is a linearized model, therefore the state variables are variations around an operation point. The MPC is instead controlling the linearized state variables and sending that to modify the initial operation point. The operation point being used is calculated by simple force balance in the hover condition, to obtain the required angular speed of the rotors to keep the quadrotor static in the air, which is approximately $360$ radians per second. That way, the control signal vector going to the quadrotor simulator is built as follows:

\begin{equation*}
\mathbf{u} = \underbrace{\mathbf{\bar{u}}}_\textrm{operation point} + \underbrace{\Delta \mathbf{u}}_\textrm{controlled by MPC}
\end{equation*}

In order to adjust to this, the constraints in the input were also modified, through a displacement of the operation range for this variation of the angular speed. The original operation range is taken from the experiments realized by Sun \cite{ref:YueSun2012}, which is $130 \leq \omega \leq 500$, in radians per second. In order to be congruent with that range, the variation of the angular speed is limited between $-230 \leq \Delta \omega \leq 140$ around the operation point, again in radians per second.\\ 

Regarding the tuning of MPC, the parameters available to adjust are the horizons and the weight matrices in the cost function. The selection of a proper prediction horizon for any MPC application is dependant on the dynamics of the system that is being controlled.  This choice is of great influence in the size of the optimization problem to solve, and therefore in the computational power required to provide deterministic operation. If the horizon is too short, the prediction won't give information about future control signals and might create unstability in the controller \cite{ref:Gabrielsson2012}. On the other hand, if the horizon is too long, the optimization problem to solve could be too large to solve in each time sample. In this particular case, since the model being used for prediction is a linearized version, a very long prediction in the future might mean moving too far away from the operation point, which will cause erratic predictions due to nonlinearities. \\

With the weight matrices, the procedure is also made in a trial and error fashion. A good initial guess for the quadrotor model is taken from previous implementation parameters \cite{ref:Bouffard2012}.

