\makeatletter
\renewcommand*\@makechapterhead[1]{%
  \vspace*{50\p@}%
  {\parindent \z@ \raggedright \normalfont
    \ifnum \c@secnumdepth >\m@ne
  %      \huge\bfseries \@chapapp\space \thechapter
  %      \par\nobreak
 %       \vskip 20\p@
    \fi
    \interlinepenalty\@M
    \Huge \bfseries \thechapter \space #1\par\nobreak
    \vskip 10\p@
  }}
\renewcommand*\@makeschapterhead[1]{%
  \vspace*{50\p@}%
  {\parindent \z@ \raggedright
    \normalfont
    \interlinepenalty\@M
    \Huge \bfseries  #1\par\nobreak
    \vskip 40\p@
  }}
\makeatother

\setlength{\parindent}{0cm}

\usepackage[nottoc]{tocbibind}
\usepackage[labelfont=bf]{caption}
\usepackage{epsfig}
%\usepackage{times}
\usepackage{multirow}
\usepackage{float}
\usepackage{subcaption}
\usepackage{longtable}
\usepackage[nomessages]{fp}
\usepackage{tabularx}
\usepackage{comment}
\usepackage{KTHEEtitlepage}
\usepackage[hmargin=2.5cm,vmargin=2.5cm]{geometry}
\usepackage{fancyhdr}			% contains customizeable headers and footers
\pagestyle{fancy}
\usepackage{wrapfig}
\usepackage{booktabs}
\usepackage{graphicx}
\usepackage{fancyref}
\usepackage{amsmath,amssymb}   % Contains mathematical symbols
%\usepackage[english]{babel}
\usepackage[T1]{fontenc}
\usepackage{pslatex}
\usepackage[utf8]{inputenc}
\usepackage{natbib}  %Nice author (year) citations
%\usepackage[round,authoryear]{natbib}  %Nice author (year) citations

%\addtolength{\textheight}{-8mm}% Adds it to the text height
%\headheight = 12mm


%\addtolength{\topmargin}{-30mm}% Removes 30mm from the top margin
%\addtolength{\textheight}{30mm}% Adds it to the text height

\usepackage{hyperref}
\hypersetup{
    colorlinks=true,       % false: boxed links; true: colored links
    linkcolor=black,          % color of internal links (change box color with linkbordercolor)
    citecolor=black,        % color of links to bibliography
    filecolor=black,      % color of file links
    urlcolor=blue           % color of external links
}
\usepackage{array}
\usepackage[final]{pdfpages}
\usepackage{nomencl}
\usepackage[num,english]{isodate}
\usepackage{acronym}
\makenomenclature
\renewcommand{\nomname}{List of Acronyms}
\newcommand*{\nom}[2]{#1\nomenclature{#1}{#2}}





















\newcommand{\newPar}{\\[11pt]}


\newcommand{\surname}{Diaz}
\newcommand{\firstname}{Rene}
\newcommand{\mail}{radm@kth.se}
\newcommand{\schoolDepartment}{Department of Machine Design}
\newcommand{\schoolName}{Royal Institute of Technology}
\newcommand{\schoolAcronym}{KTH}
\newcommand{\reportTitle}{A library on the Robot Operating System (ROS) for Model Predictive Control implementation}
\newcommand{\reportSubtitle}{}
\newcommand{\reportDate}{December 2014}
\newcommand{\reportRefNr}{}
