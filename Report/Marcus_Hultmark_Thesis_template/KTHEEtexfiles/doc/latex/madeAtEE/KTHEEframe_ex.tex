%%%%%%%%%%%%%%%%%%%%%%%%%%%%%%%%%%%%
% s3frame_ex.tex
%
% Example on how to use the KTHEEframe style to produce slides with an
% KTH/EE logotype to the left. 
%
% Mats Bengtsson,  7/8 2006
%%%%%%%%%%%%%%%%%%%%%%%%%%%%%%%%%%%%
\documentclass[a4]{seminar}

\usepackage{KTHEEframe}

\begin{document}
\begin{slide}
  \heading{Introduction}
  This is a first slide of the presentation.
  
\end{slide}

\begin{slide}
  \heading{Next Slide}
  
  This package is based on the \texttt{seminar} document class, which
  provides only very basic support for slides. In addition to the KTH
  logotype, the \texttt{KTHEEframe} package also adds a \verb+\heading+
  command to specify a heading for each slide.

  Note that it has to be processed with latex + dvi2pdf (or latex +
  dvips + ps2pdf). pdflatex or dvipdfm will not work.

\end{slide}


\end{document}