% !TeX spellcheck = en_US


%\todo[inline]{deptno etc??}

\noindent \begin{tabular*}{1.0\textwidth}{|@{} p{0.9835\textwidth}|}
\hline
\noindent \begin{tabular*}{1.0\textwidth}{p{0.97\textwidth}}
\textcolor{white}{.}\\[-10pt]
\end{tabular*}
\noindent \begin{tabular*}{1.0\textwidth}{p{0.24\textwidth} p{0.69\textwidth}}
\multirow{3}{*}{\includegraphics[scale=1]{./Abstract/KTH-pic-en}} & \begin{center}Master of Science Thesis MMK 2014:90 MDA 465\end{center}\\[-20pt]

& \begin{center}\reportTitle\end{center}\\[-20pt]
& \begin{center}Rene Diaz \end{center}\\ 
\end{tabular*}
\noindent \begin{tabular*}{1.0\textwidth}{p{0.24\textwidth}|p{0.33\textwidth}|p{0.33\textwidth}}
\hline
{ \footnotesize Approved:} & { \footnotesize Examiner:} & { \footnotesize Supervisor:}\\
2014-11-14 & Lei Feng & Bengt Eriksson \\
\hline
& { \footnotesize Commissioner:} & { \footnotesize Contact person:}\\
& Universidad Sim\'{o}n Bol\'{i}var & Carlos Mastalli \\ \hline
\end{tabular*}
\end{tabular*}\\

\justify
\textbf{Abstract}\\

Model Predictive Control is a receding horizon control technique that is based on making predictions in the future for a determined number of steps, using a model of the system to be controlled. This thesis report is centered around Model Predictive Control (MPC) and its application. In this thesis, there are two main goals: firstly, is the development of a software structure that uses the properties of Object Oriented Programming (OOP) and the Robot Operative System (ROS) to ease the use of MPC applications. Secondly, the use and verification of the capabilities of MPC controllers in plants with fast dynamics, such as the quadrotor. A linearized model of the quadrotor is developed for the controller to perform the predictions, and the non-linear version is used to make a numerical simulator to test the application. The MPC software structure works as it successfully integrates information from the classes that represent the model and optimization method to solve the quadratic problem. The resulting MPC controller shows a good response when following simple trajectories in the presence of simulated noise. However, when more complex trajectories are used, a considerable offset from the reference is obtained. Such behavior mostly caused by the use of a very limited model, which demonstrates the considerable sensibility of the controller to the accuracy of the used model.

